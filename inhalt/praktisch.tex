\thispagestyle{empty}
\section{Praktische Durchführung}
\subsection{Hardware}
\subsection{Docker}
\subsection{Realisierung}
Im Linux-File-System ist es möglich cgroups über PID's ausfindig zu machen und die verwendeten Ressourcen auszugeben. Was mich auf die Idee brachte, einen neuen Docker-Container zu erstellen, in diesem einen Prozess mit einem einfachen malloc() befehlt in einer Endlosschleife auszuführen und die menge des allokierten Speichers der cgroup auszugeben. 

Über das online user interface war es einfach einen Docker-Container schnell zu erstellen. Ressourcen Limits konnten ebenfalls vor dem starten des Containers eingestellt werden. Für den ersten Start des Containers benötigen wir allerdings noch keine Begrenzung des Speichers, da es in erster Linie um die Auswirkung des ausgeführten C-Programms auf die eigene cgroup geht. Der verwendete C-Code den der Prozess ausführt ist in Abbildung (C-Code While1) screenshot)) Ein selbst erstelltes Skript verwendet die beim Start erzeugte PID, findet dadurch die cgroup in der der Prozess ausgeführt wird und schreibt die Messdaten in ein txt-File. Die ausgegebenen Messdaten wie Clock-Time in NS und Speicherverbrauch in Bytes werden für die Auswertung verwendet.

\subparagraph{Erwartungshaltung}

(Siehe Abbildung (cgroup ohne limit noch nicht erstellt)

Als nächstes wäre es interessant zu sehen wie die Cgroup verläuft, wenn man ein Speicherlimit festlegt, das nicht überschritten werden soll (Hard-Limit). Um dieses Limit zu erstellen, wurde im User-Interface lediglich das Ressourcenlimit auf [X]MegaByte gesetzt und das in Abbildung(X) vorgestellte Programm nochmals ausgeführt.

\pagebreak