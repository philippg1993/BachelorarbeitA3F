\thispagestyle{empty}
\section{Isolation}
Dank der Erkenntnisse im Praktischen Abschnitt kann in dieser Section auf den Isolations unterschied der Virtualisierungsmethoden eingegangen werden.

\subsection{Container}
In dieser Arbeit wird ausschließlich Linux verwendet, weshalb auf die Eigenschaften im Linux Kernel eingegangen wird. In einem anderen Betriebssystem können einzelne Punkte unterschiedlich implementiert sein. Die Speicher Verwaltung im Container System übernimmt das darunter liegende Betriebssystem. Weil viele Anwendung Speicher anfragen, bevor sie Ihn verwenden, wurde der Linux kernel mit einer "Over commit" lösung ausgestattet um Ressourcen effizient zu nutzen. Over commit bedeutet einfach übersetzt Überschulden. Mit dieser Methode wird speicher virtualisiert indem anfragen von Anwendungen immer akzeptiert werden. Dadurch vergibt der Kernel mehr Speicher als auf der Hardware tatsächlich vorhanden ist. Die Anwendungen bekommen genau so viele Hardwareressourcen, die sie Aktuell verbrauchen. Mit diesem System kann es vorkommen, dass mehr Ressourcen benötigt werden als Real vorhanden sind. 

\subsection{Virtuelle Maschine}
Das Ziel der Prozessisolation ist es, zu verhindern, dass Container untereinander beeinflussbar sind. Dieses Unterfangen ist mit Hilfe der sogenannten "namespaces" möglich. Die Ressourcen Isolation gewährleistet, dass Prozesse nur einen zugeteiltet anteil einer verfügbaren Ressource verwenden können, dies gelingt mit Hilfe der Cgroups.


vllt noch zu erklären!?
Virtuallisierungen
Vollstendige virtuallisierung 
Paravirtualisierung

Hyper-V
KVM

OOM Killer für Fazit


\subsubsection{Sicherheit-Container / Umschreiben!}
In den letzten Jahren unterlag die Entwicklung von Container-Systemen wie Docker, Rocket und andern einer sehr schnellen Evolution. Dabei war der Fortschritt der Technologie wichtiger als die Sicherheit hinter dem System. Es existieren zahlreiche Studien, die alle auf den Schluss kommen, dass viel Potential in der Entwickelten Technologie steckt, doch die Sicherheit noch nicht ausreichend gegeben ist.

Zwei Punkte spielen eine große Rolle. Zum einen die Sicherheit und Vertrauenswürdigkeit der Images, die den Container-Instanzen zugrunde liegen, zum anderen die oben benannten Namespaces. Die Nutzung von Namespaces stellt einen direkten Zugriff auf den Host-Kernel dar, ein Bereich zu dem sich Angreifer bei anderen Systemen erst einmal mühevoll durchkämpfen müssen.