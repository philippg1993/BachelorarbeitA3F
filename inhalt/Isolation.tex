\thispagestyle{empty}
\section{Isolation Container}

In dieser Arbeit wird ausschließlich Linux verwendet, weshalb auf die Eigenschaften im Linux Kernel eingegangen wird. In einem anderen Betriebssystem können einzelne Punkte unterschiedlich implementiert sein. Die Speicher Verwaltung von Containern übernimmt das darunterliegende Betriebssystem. Weil viele Anwendung Speicher anfragen, bevor sie Ihn verwenden, wurde der Linux-Kernel mit einer "Over Commit" Lösung ausgestattet um Ressourcen effizient zu nutzen. Over Commit bedeutet einfach übersetzt Überschulden. Mit dieser Methode wird Speicher virtualisiert indem Ressourcen Anfragen von Anwendungen vom Kernel immer akzeptiert werden. Dadurch vergibt der Linux-Kernel mehr Speicher als auf der Hardware tatsächlich vorhanden ist. Die Anwendungen selbst bekommen genauso viele Memory Pages auf der Hardware, wie sie Aktuell benötigen. Mit diesem System kann es vorkommen, dass mehr Ressourcen benötigt werden als Real vorhanden sind. 

Wenn die Ressourcen auf der Hardware knapp werden, ist Paging die erste Reaktion des Betriebssystems. Beim Paging werden Allokierte aber aktuell nicht verwendete Memory Pages vom Arbeitsspeicher auf den Massenspeicher geladen. Da ein Massenspeicher Zugriff sehr aufwändig und langsam ist, wurde diese Möglichkeit der zusätzlichen Speicher Gewinnung limitiert. Auf dem im Praxis Teil verwendeten System liegt das Limit bei 2GB. Wenn Paging nicht ausreicht, wird der "OOM-Killer" aufgerufen. Der Out of Memory-Killer beendet anhand einer Prioritätenliste die niederwertigsten Prozesse, bis ausreichend Platz geschaffen wurde. Jeder Prozess erhält bereits bei der Erstellung eine Priorisierung in Form einer Zahl, die nach Eigenschaft und Wichtigkeit die Priorität wiederspiegelt.

Bei der Erstellung von Containern wird ein in der Maximalen Größe definierter durch C-Group beschränkter und durch Namespace Isolierter Bereich geschaffen. Die aktuelle Größe des Bereichs hängt von den aktuell verwendeten Ressourcen ab und liegt im Regelfall unter der maximalen Begrenzung. Mehrere voneinander gut isolierte Container teilen sich ein durch das Betriebssystem verwaltetes Hardwaresystem. Wegen der Speicherverwaltung des Betriebssystems also des Linux Kernels, ist es möglich, dass ein Container durch einen erhöhten Speicherverbrauch oder durch eine schlechte Kalkulation des Administrators, mit Hilfe des OOM-Killers Einfluss auf einen anderen Container nehmen kann, was die Illusion der Vollständigen Isolation der Container zerstört. 

Das gerade beschriebene Szenario soll im Praktischen Teil dieser Arbeit untersucht werden. Das daraus resultierende Ergebnis wird im letzten Abschnitt mit der Hypervisor-Virtualisierungs Variante verglichen. Als Container-Basierte Virtualisierungs Lösung wurde Docker ausgewählt. Docker ist Aktuell die meist verbreitetste Softwarelösung für die Erstellung von Containern und basiert auf dem Prinzip der Linux-Container(LXC).





