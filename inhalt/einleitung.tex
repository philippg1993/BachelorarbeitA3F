\thispagestyle{empty}
\section{Einleitung}
In der Industrie geht es meist darum, Kosten einzusparen und effizient zu arbeiten. Moderne Hardwaresysteme werden häufig nur zu einem geringen Prozentsatz ausgelastet. Um zusätzliche Aufgaben zu erfüllen, muss weitere Hardware gekauft, installiert und unterhalten werden, was mit unnötigen Kosten verbunden ist. Um Systeme effizienter zu nutzen, greift man seit einiger Zeit zur Virtualisierung von Hardware. So können Hardwareressourcen wie Speichermedien, \ac{CPU}, \ac{I/O} und Netzwerk in kleinere Einheiten unterteilt werden, die verschiedene Aufgaben auf einem System ausführen. Somit erübrigt sich der Kauf weiterer Hardware und die Systeme werden effizienter genutzt. 

Vor allem in der Automobilindustrie geht es bei der Entwicklung von Serienfahrzeugen um möglichst effizient ausgenutzte Räume und Ressourcen. Bereits Centbeträge, die pro Fahrzeug eingespart werden, schlagen bei einer millionenfachen Produktion zu Buche.

Diese Arbeit wird im Rahmen des Projektes \ac{A3F} erstellt. Ziel des Projektes ist, eine moderne und zukunftsorientierte Lösung für das Problem von immer mehr einzelnen Steuergeräten zu finden. Auch die Ausfallsicherheit der einzelnen Steuergeräte soll erhöht werden. Der Ansatz liegt darin weniger, dafür leistungsfähigere, Steuergeräte zu integrieren, auf denen voneinander isolierte Prozesse laufen. Das wird durch den Einsatz von Virtualisierungstechniken erreicht. 

Diese Arbeit untersucht die Isolation von Container- und Hypervisor-basierten Virtualisierungstechniken. Einige Prozesse sind sicherheitskritisch und dürfen unter keinen Umständen von anderen Prozessen beeinflusst werden. Beispielsweise darf in einem Automobil das Bremssystem nicht ausfallen, wenn die Innenraumbeleuchtung ausfällt. In der vorliegenden Arbeit wird die mögliche Wechselwirkung zwischen isolierten Prozessen betrachtet und eventuelle Lösungsansätze aufgezeigt.


\pagebreak