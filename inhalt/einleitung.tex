\thispagestyle{empty}
\section{Einleitung}
In der Industrie geht es meist darum wie man Kosten einsparen kann. Die Hardware Systeme in Firmen werden häufig nur zu einem geringen Prozentsatz ausgelastet. Neue Hardware für zusätzliche Anforderungen ist sehr teuer. Wenn es nur eine Möglichkeit geben würde auf einem System mehrere voneinander isolierte Umgebungen zu schaffen und die schon vorhandenen Kapazitäten auszunutzen, könnten Kosten durch neue Anschaffung, Platz und sogar Strom gespart werden. Vorallem in der Automobilindustrie geht es bei der Entwicklung von Serienfahrzeugen um möglichst effizient ausgenutzte Räume und Ressourcen. Bereits Cent Beträge die Pro Fahrzeug eingespart werden können, schlagen bei einer Millionenfachen Produktion zu Buche. 

Container als leiftgewichtiger Virtualierung bietet ein 

gegenüberstellung Effizienz Isolierung 
Hardwareaufwand
Ausfallsicherheit 

\subsection{Motivation}
\subsection{Projektbeschreibung}
\subsection{Ziele}

Effiziente ressourcenverteilung 

Überlastung der Hardware
geld sparen 




\pagebreak