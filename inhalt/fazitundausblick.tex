\thispagestyle{empty}
\section{Fazit und Ausblick}

Im Rahmen dieser Arbeit wurden Prozess-Isolationsmechanismen bei Containern und virtuellen Maschinen analysiert und verglichen. Container sind so konzipiert, dass sie die zur Verfügung stehende Hardware dynamisch und dadurch effizient nutzen. Bei Containern wird die Over-Commit-Lösung im Ressourcen-Management verwendet. Das bedeutet, bei der Container-basierten Virtualisierung wird die zur Verfügung stehenden Hardware virtuell vergrößert. Im Gegensatz dazu vergibt ein Hypervisor nur die ihm zur Verfügung stehenden Ressourcen. Er virtualisiert die Ressourcen nicht selbst, sondern erstellt durch die virtuelle Teilung der Hardware mehrere Partitionen der Rechnerarchitektur. 

Die dargestellten Versuche haben gezeigt, dass Container durch die Over-Commit-Lösung nicht perfekt isoliert sind. Durch zu hohe Auslastung der Ressourcen können die Grenzen der Container umgangen werden. Die Over-Commit-Lösung ist somit nicht optimal für die isolierte Virtualisierung geeignet, jedoch für die Dynamik und die Effizienz unerlässlich.

Um wichtige Prozesse zu schützen, könnte im Falle eines OOM-Killer Aufrufs die Priorität sicherer Container, bei denen die Images und ausgeführten Programme überprüft werden, manuell erhöht werden. Zugleich kann für unkontrollierte und weniger wichtige Container die Priorität verringert werden.

Eine Möglichkeit, um die Isolation zwischen Containern zu verstärken und die Prozesse abzusichern, ist die Veränderung der Over-Commit-Implementierung im Linux-Kernel. Die Verbesserung der Isolation bedeutet in diesem Fall eine Verschlechterung der Effizienz. 

Die Kombination aus virtueller Maschine und Container kann eine Lösung in der Automobilindustrie der Zukunft sein. Je nach Sicherheitsanforderung werden durch verschiedene Over-Commit-, OOM-Killer- und Prioritätenvergabe-Lösungen im modifizierten Linux-Kernel implementiert, die auf einem Hypervisor mit Minimalausführung eines Betriebssystems laufen. Bei sicherheitskritischen Anwendungen wie Airbag, Bremsen oder Lenkrad wird ein Linux-Kernel so umgebaut, dass die Over-Commit Variante wie bei Hypervisor komplett herausgenommen wird und jede Anwendung seinen eigenen Ressourcenbereich zur Verfügung gestellt bekommt. Dadurch sind Container untereinander stärker abgesichert. Bei den Blinkern oder der Innenraumlüftung kann eine Linux-Kernel-Modifikation mit geringem Over-Commit ausgestattet sein und einer zusätzlichen Priorisierungsvariante im Falle eines Überlaufs. Bei Anwendungen wie Innenraumbeleuchtung, Autoradio oder Navigationssystem ist die ursprüngliche Over-Commit-Lösung implementiert, um die vorhanden Ressourcen am effizientesten zu nutzen.

In dieser Arbeit wurde nicht auf alle aufgetretenen Seiteenffekte eingegangen. Eine weiterführende Arbeit könnte sich mit der Deckelung von Ressourcen durch Cgroup genauer beschäftigen. Die tatsächliche Menge an allokiertem Speicher auf der Hardware eines Prozesses, kann durch das Schreiben einer definierten Zahlenfolge auf den Arbeitsspeicher überprüft werden. Die Speicherbereiche der Cgroup auf der Hardware und die Memory-Pages auf dem Massenspeicher werden ausgelesen und mit der geschriebenen Folge verglichen. Wenn ein Teil der Zahlenfolge fehlt, ist die durch Cgroup dargestellte Limitierung nicht eingehalten worden. 

\pagebreak